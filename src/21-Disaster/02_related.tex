\section{Related work} \label{sec:Related work}

Research communities are paying more and more attention for \ac{EES},
	this particular attention is mainly due to the interest of building companies to deploy smart sensors in their construction sites.
The main factors of a good deployment of such systems is about the quality of the network that makes all emergency agents communicate and collaborate to make good decisions.
In another hand,
	the need of population of new tools to deal with the permanently dynamic changes of pedestrian and car paths in their city,
	becomes very challenging.
For example,
	Qiu, et al \cite{qiu_ergid_2016} proposed an emergency response system with sensor nodes for real time application
	they proposed an iterative method for delay optimizing with the lossless transmission,
	authors tried to find a trade-off between the increased energy consumption and the reduced network lifetime.
The framework proposed in by Al Turjman, et al \cite{al-turjman_cognitive_2019} is a multi hop routing method for disaster management in real time.
To achieve a higher efficiency,
	the system, they propose, is designed for using limited multiple hops with the help of left over energy may sometimes result with loss of data due the exhausted energy level.


%%%%%%%%%%%%%%%%%%%%%%%%%%%%%%%%%%%%%%%%%%%%%%%%%%%%%%%%%%%%%%%%%% Research categories %%%%%%%%%%%%%%%%%%%%%%%%%%%%%%%%%%%%%%%%%%%%%%%%%%%%%%%%%%%%%%%%%%%%%

%%%%%%%%%%%%%%%%%%%%%%%%%%%%%%%%%%%%%%%%%%%%%%%%%%%%%%%%%%%%%%%%%%%% Related work position %%%%%%%%%%%%%%%%%%%%%%%%%%%%%%%%%%%%%%%%%%%%%%%%%%%%%%%%%%%%%

%All previous work didn't take into consideration the topological aspect of interactions to measure social vulnerabilities of users.
%The closest study to our approach is that presented in \cite{zeng_trustaware_2014}.
%However,
%this solution doesn't study the impact of having interactions with vulnerable users.
%In this paper,
%we study the impact of trusting vulnerable users in preserving the privacy of all users in the communication network.


