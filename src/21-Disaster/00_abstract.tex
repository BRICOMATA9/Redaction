\begin{abstract}

% Extreme events and disasters resulting from climate change or other ecological factors are difficult to predict and manage.
% Current limitations of state-of-the-art approaches to disaster prediction and management could be addressed by adopting new unorthodox risk assessment and management strategies.
% The next generation Internet of Things (IoT),
% Wireless Sensor Networks (WSNs), 5G wireless communication,
% and big data analytics technologies are the key enablers for future effective disaster management infrastructures.
% In this chapter,
% we commissioned a survey on emerging wireless communication technologies with potential for enhancing disaster prediction,
% monitoring,
% and management systems.
% Challenges,
% opportunities,
% and future research trends are highlighted to provide some insight on the potential future work for researchers in this field.

% Cause of the problem 1 ------------------------------------------------
The number of building sites increased considerably these few last years.
Emerging countries invest more and more in building and construction sites to overcome the need of citizens like education,
	residential and administration buildings.
Such sites should be equipped with evacuation systems to allow bricklayers and other persons to be evacuated quickly.
% Cause of the problem 2 ------------------------------------------------
Furthermore,
	evacuation systems should communicate with each other to coordinate evacuation emergency of people.
% The problem -----------------------------------------------------------
One of the most important problem of such systems is the need of a powerful network technology with sufficient range that could make communication in sites of several kilometers squares.
The lack of such mechanisms make the life of billion citizens in a danger.
% Motivation to solve this problem --------------------------------------
This problem could be addressed by building a novel evacuation emergency application in \ac{IoT} devices with \ac{LPWAN}.
One of the known \ac{LPWAN} networks is \ac{LoRa} network.
Our work is motivated by the deployment of an \ac{EES} in building sites.
% Goal ----------------------------------------------------------------
Our goal is to measure the \ac{QoS} of IEEE.802.15.4 and LoRa technologies and adapt their configurations to the emergency situation of the site.

% % Challenges ----------------------------------------------------------

% The difficulty is mainly due to
% % Approach 1 ----------------------------------------------------------
% To address this problem,

% % Approach 2 ----------------------------------------------------------
% Particularly,

% % Validation ----------------------------------------------------------
% We validate our approach by

% % Results1 ----------------------------------------------------------
% Simulation results show that

% % Results2 ----------------------------------------------------------
% Furthermore,

\end{abstract}


