\section{Introduction \cite{joseignacio_contiki_2016}} \label{sec:Introduction}


\subsection{Problem Statement}


From the start of the computer networks,
	to the mobile applications nowadays,
	the amount of information shared has been constantly increasing.
We have all kind of devices,
	from the big servers in datacenters,
	the TVs at home,
	mobile phones,
	car sensors,
	...
The Internet Of Things (IOT) refers to the idea of connecting all the “things”to the Internet.
By “things”,
	it refers to any ordinary object that can be useful getting information.
These “things” should be connected by an embedded device,
	capable to connect to the Internet in one side,
	and get information from the “thing” on the other.

\subsection{Background}

\subsection{Purpose (Goal)}

The first objective of this thesis is to document the capabilities of the Z1 motes and the Contiki OS for the IOT,
	by building applications to gather data from sensors and the network capabilities both from the motes and the OS.
Secondly,
	to build an application using the Z1 motes,
	and the Contiki OS,
	using COAP(Constrained Application Protocol) and 6LoWPan(IPv6 over Low power Wireless Personal Area Networks) to retrieve information from the motes,
	and connect them to Sentilo,
	an open source sensor and actuator platform.

\subsection{Limitations}

There has been an increased research and development for the Smart Cities.
The smart cities objective is to gather information from the city,
	to enhance quality and performance of urban services,
	to reduce costs and resource consumption,
	and to engage more effectively and actively with its citizens.
This project is intended to approach two goals,
	to be used as a starting point for anyone who wants to use the Contiki OS with the Z1 motes,
	and to build a simple application for the Smart Cities,
	to collect data from sensors,
	and sending it to an information center.

\subsection{Method}

This document is divided in two parts.
First,
	the description of the main tools used to create the experimental environments.
A description of the Contiki OS,
	the Z1 motes,
	and the protocols used to communicate,
	IEEE 802.14.5, 6LowPAN and CoAP.
The a brief description of the sensor data collector Sentilo Secondly,
	a description of the environment setup,
	and an explanation of how it works.
Finally,
	conclusions and future work is presented.



% The structure 
This paper is organized as follows.
Section \ref{sec:Related work} elucidates summary of related works.
% Section \ref{sec:Background} provide the required background.
In section \ref{sec:Approach}, we propose our ... to ....
Section \ref{sec:Experimentation} evaluates the performance of our ... in terms of packet delivery ratio,
	throughput,
	and power consumption.
% Our findings are presented in section \ref{sec:Results}.
Section \ref{sec:Conclusion} concludes the article and gives some ideas for future work.
