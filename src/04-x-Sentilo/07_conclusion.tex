\section{Discussion} \label{sec:Conclusion}

The Contiki OS,
	collects all the technologies needed for the development of centralized data collectors,
	for the sensors.
This platform combined with Sentilo,
	creates a real application platform,
	to be able to deploy in several possible real environments.
The main advantages of Contiki,
	are how easy is to create code,
	and generate concurrent scenarios inside the same mote,
	being able to have a web server at the same time a root node of a WSN is running,
	without complexity.
At the same time ,
	the application level library as COAP,
	with the complete examples of this libraries,
	makes this system a powerful and versatile tool.
A disadvantage of this platform,
	is the lack of documentation and examples,
	outside the inner code.
There's a lot of time and test to make,
	for a more complex application.
Secondly,
	the Sentilo platform,
	is an easy to install,
	use and program applications with.
It has a wide set of options and tools,
	that need to be understand carefully for a rich application that uses all the functionalities properly.
The combinations of both,
	makes a good,
	simple and potentially improvable scenario,
	for centralize data collection.

\subsection{Future lines of work}

There are some future lines of work in this experimental environment: 

1. Test the CoAP server in the new release of Contiki.
Contiki 3.0 A new release of Contiki was released in September 2015,
	with some changes and improvements overall,
	specially with CoAP.
The new release supports CoAP 18.

2. A Java connector with a dynamic network.
The Java connector finds the motes in a stable WSN,
	if a node is missing or replaced,
	it needs a manual interaction to find all the motes gain,
	by erasing all the network,
	and start to find all the motes again.
Besides,
	the protocol handling the routes,
	is IP and the protocol handling the links is RPL.
The IP routes in the border router expire every certain time,
	that means that if a mote is missing,
	a route is still present for a certain time,
	even if the RPL is aware of the missing mote.
As a possible solution,
	there are repairing route methods in CoAP that are used to repair the broken links between nodes.




