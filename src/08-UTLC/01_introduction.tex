\section{Introduction} \label{sec:Introduction}

% 1) Motivation
According to the French agency of statistics on roads' accidents, 12\%  of them happen in intersections caused by the non-compliance to traffic rules. 23\% of them led to hospitalization in which 14\% are fatal \cite{Routiere2015}.
Urban Traffic Light Control (UTLC) are one possible solution to regulate vehicle flows at intersections.
However,
	a static cycle of traffic lights (or lights signs) has a direct impact on traffic jams,
	particularly when an emergency vehicle must cross through as quickly as possible.
A long period at red or green light might impact the fluidity of the city traffic.

% 4) Proposition + what is in our work that differ with existing solutions
The Internet of Things (IoT) and Everything (IoE) can be a solution to adapt traffic light control to traffic density.
It allows heterogeneous connected objects,
	\textit{e.g.} Zigbee,
	LoRa,
	SigFox,
	ITS-G5,
	to interact and exchange sensed data on roads,
	vehicles,
	pedestrians presence,
	time of leaving house,
	etc  \cite{Perera2014}.
Therefore,
	connecting heterogeneous infrastructures following a Device-to-Device approach is possible through upper layers or an intermediate Cloud platform.
Wireless Sensor Networks (WSNs) are the source of these data,
	and the Cloud is the remote entity that collects them.
Fog and Edge computing have been proposed to address the low latency of IoT applications by efficient resource distribution and a local processing.
Fog computing leverages Edge devices and remote and private Cloud resources with distributed data processing.
It provides the advantage to process data closer to the source and thus mitigate latency issues and reduce network congestion.
However,
	constructing a real IoT Fog Computing is costly for evaluation while the environment has to be controllable for experiments \cite{Dastjerdi2016}.
However,
	in our work we would implement our Edge computing with a remote Cloud data gathering and deal with latency through QoS protocol.
Remote Cloud offers the possibility to integrate new services.
For example,
	we can deploy sensors to measure noise or air pollution via traffic signs or roads.

Our objective is to model,
	prototype and evaluate the Quality of Service (QoS) of an IoT solution for a traffic light control system.
Modeling  of traffic light states control is essential to avoid incoherent situations.
Number of models based on Petri Nets (PN) have been proposed in \cite{difebbraro_trafficresponsive_2006} and \cite{huang_modular_2014}.
Their main drawbacks are the limitation to the structural analysis of state transitions and the lack of verification.
However,
	our design is based on UPPAAL (UPPsala and AALborg Universities) \cite{david_uppaal_2015} timed automata for design and verification of coherent state of cross road's traffic light.
It specifies a graph of states with clocks and data variables.
To implement our Urban Traffic Light Control based on an IoT network architecture (IoT-UTLC),
	we setup a real IEEE 802.15.4 WSN with devices that can act as actuators and sensors.
Small traffic light signs are driven by a Border Router (BR) device  to a sink node which is a gateway to the Internet.
This BR is connected to a host computer (or sink) also connected to an IoT Cloud platform.
WSN devices forward their data to the IoT Cloud through this sink which defines required levels of QoS based on Message Queuing Telemetry Transport (MQTT) protocol \cite{Al-Fuqaha2015}.
The collected data can be transmitted to different devices such as sink,
	BR or wireless sensor/actuator devices.
When WSN devices detect the arrival of high priority vehicles,
	sensed data are routed to the IoT Cloud.
Then,
	the sink node takes a decision to change the light's state and forward generated messages to actuator devices through BR with a high level of QoS.
Thanks to IPv6 over Low power Wireless Personal Area Network (6LoWPAN) \cite{chalappuram_development_2016},
	our WSN is energy-efficient and IPv6 compatible.

%%% Objective
%The aim of this work is to prototype our solution and to get a persistent connection between traffic lights and the IoT Cloud Platform.
%By using specific technologies,
%we would like to get a minimum latency when sending and receiving packets   % Results
%Traducing those objectives,
%we want to obtain proof that using MQTT in this non-reliable network which is the Internet and its QoS levels,
%would be more efficient than the actual situation with time optimization and delivering and processing guarantees.

% 6) Organisation
This paper is organized as follows.
In Section \ref{Sec:Related_Works},
	we review related work.
Section \ref{sec:Use Case and Model Design} reports the design of our prototyping solution.
We describe the use case defined with the design model.
Section \ref{sec:Prototyping} defines our prototype (IoT Testbed),
	giving our specifications and discussing on our choices of technologies and protocols.
Finally,
	Section \ref{sec:Results} presents the obtained results that show the usefulness and the best practice for MQTT.
