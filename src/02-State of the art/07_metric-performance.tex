\section{Radio frequency performance}

\subsection{Power Level (dB)}

The dB measures the power of a signal as a function of its ratio to another standardized value. The
abbreviation dB is often combined with other abbreviations to represent the values that are compared.
Here are two examples:
\Itemize{
	\item dBm—The dB value is compared to 1 mW.
	\item dBw—The dB value is compared to 1 W.
}

\Equation{power}{Power (in dB) = 10 * log(10) (Signal/Reference)}

Where:
\Itemize{
	\item log(10) is logarithm base 10.
	\item Signal is the power of the signal.
	\item Reference is the reference power.
}

For example, if you want to calculate the power in dB of 50 mW:

Power (in dB) = 10 * log(10) (50/1) = 10 * 1.7 = 17 dBm

\subsection{Receive Signal Strength Indicator RSSI}

Receiver sensitivity is defined as the minimum signal power level with an acceptable Bit Error Rate (in dBm or mW) that is necessary for the receiver to accurately decode a given signal.
This is usually expressed in a negative number depending on the data rate.
For example an Access Point may require an RSSI of at least negative -91 dBm at 1 MB and an even higher strength RF power -79 dBm to decode 54 MB.

\subsection{Signal to Noise Ratio SNR}

Noise is any signal that interferes with your signal.
Noise can be due to other wireless devices such as cordless phones,
	microwave devices,
	etc.
This value is measured in decibels from 0 (zero) to -120 (minus 120).
Noise level is the amount of interference in your wireless signal,
	so lower is generally good for WLAN deployments.
Typical environments range between -90dBm and -98dBm with little ambient noise.
This value may be even higher if there is a lot of RF interference coming in from other non-802.11 devices on the same spectrum Signal to Noise Ratio or SNR is defined as the ratio of the transmitted power from the AP to the ambient (noise floor) energy present.
To calculate the SNR value,
	we add the Signal Value to the Noise Value to get the SNR ratio.
A positive value of the SNR ratio is always better.
For example,
	say your Signal value is -55dBm and your Noise value is -95dBm.
The difference of Signal (-55dBm) + Noise (-95dBm) = 40db—This means you have an SNR of 40.
Note that in the above equation you are not merely adding two numbers,
	but are interested in the “difference” between the Signal and Noise values,
	which is usually a positive number.
The lower the number,
	the lower the difference between noise and transmitted power,
	which in turn means lower quality of signal.
The higher the difference between Signal and Noise means that the transmitted signal is of much higher power than the noise floor,
	thereby making it easier for a WLAN client to decode the signal.


\subsection{Signal Attenuation}

Signal attenuation or signal loss occurs even as the signal passes through air.
The loss of signal strength is more pronounced as the signal passes through different objects.
A transmit power of 20 mW is equivalent to 13 dBm.
Therefore if the transmitted power at the entry point of a plasterboard wall is at 13 dBm,
	the signal strength will be reduced to 10 dBm when exiting that wall.
Some common examples are shown in Table 10-5.