\subsection{Technical (Mail/Web)}

%\subsubsection{Network layer}

%Client and Server IP addresses are the 2 most important fields contained in the Internet Protocol (IP) headers.
%Client IP address identifies a particular user’s home gateway.
%The Server IP address,
%	in its turn,
%	contains information about customers enjoying Web services.

%To prevent this information leakage,
%	users usually exploit anonymity network tools (TOR).
%Tor is a software that can be used for anonymization of Internet communication,
%	by routing traffic through several layers of encrypted tunnels \cite{Dingledine:2004}.

%The popular technique used to \textbf{preserve privacy} in IP-based network data is IP address anonymization.
%Several algorithms can be exploited to this end,
%	with the most popular being Crypto-PAn \cite{xu_prefix-preserving_2002},
%	a cryptography-based sanitization tool.
%However,
%	IP address anonymization still poses several challenges and questions \cite{le_blond_towards_2013}.

%\subsubsection{Application layer}

Nowadays, the Web converged over two main protocols,
	namely HTTP/HTTPS and SMTP/SMTPS
Beside these,
	DNS still has a central role for reaching almost any Web server.

%%An Effective Defense Against Email Spam Laundering
%\subsubsection{Mail (SMTP)}

Many anti-spam techniques have been proposed and deployed to counter email Spam from different perspectives.
Based on the placement of anti-Spam mechanisms,
	these techniques can be divided into two categories: recipient-oriented and sender-oriented.

\subsubsection{Recipient-oriented Techniques}

This class of techniques either
	(1) block/delay email Spam from reaching the recipient’s mailbox or
	(2) remove/mark Spam in the recipient’s mailbox.
Due to the flourish of techniques in this category,
	we further divide them into content-based and non-content-based sub-categories.

\paragraph{Seder-based reputation}

%		The techniques in this sub-category detect and filter Spam by analyzing the content of received messages,
%			including both message header and message body.

%		\textbf{Email address filters:} Whitelists and blacklists are the 2 email address filters largely used in this area.
%		Whitelists are all acceptable email addresses and blacklists are the opposite.
%		Blacklists can be easily broken by forging new email addresses,
%			but using whitelists alone makes the acceptable email enclosed.
%		Garriss et al. \cite{garriss_re:_2006} developed a new whitelisting system,
%			which can automatically populate whitelists by exploiting friend-of-friend relationships among email correspondents.

%		\textbf{Heuristic filters:} The features that are rare in normal messages but appear frequently in spam,
%			such as non-existing domain names and spam-related keywords,
%			can be used to distinguish spam from normal email.

%		\textbf{Machine learning based filters:} Since spam detection can be converted into the problem of text classification,
%			many content-based filters utilize machine-learning algorithms for filtering spam.
%		As these filters can adapt their classification engines with the change of message content,
%			they outperform heuristic filters.
%		These techniques attempt to generate on a set of samples with an acceptable error rate.

%\paragraph{Non-Content-based Techniques}

%		Non-content spam characteristics are used in this sub-category,
%			such as source IP address,
%			message sending rate,
%			and violation of SMTP standards,
%			to detect unwanted email.

		\textbf{DNSBLs:} DNSBLs are distributed blacklists,
			which record IP addresses of spam sources and are accessed via DNS queries.
		When an SMTP connection is being established,
			the receiving MTA (Mail Transfer Agent) can verify the sending machine’s IP address by querying the subscribed DNSBL.
		Even DNSBLs have been widely used,
			their effectiveness and responsiveness \cite{jung_empirical_2004, ramachandran_can} are still under study.

\paragraph{Seder-based authentication}

		\textbf{MARID:} MARID (MTA Authorization Records In DNS) is a class of techniques to counter forged email addresses by enforcing sender authentication.
		MARID is also based on DNS and can be seen as a distributed whitelist of authorized MTAs.
		Multiple MARID drafts have been proposed,
			some of them (SPF and DKIM) are deployed in real world \cite{spf:_2018, BibEntry2014Dec}.
			PGP and S/MIME are also

		\textbf{Challenge-Response (CR):} CR is used to keep the merit of whitelist without losing important messages.
		To add a sender email address in the whitelist,
			senders are requested a challenge that needs to be solved by a human being.
		After a proper response is received,
			the sender’s address can be added into the whitelist.

		\textbf{Tempfailing:} Tempfailing \cite{twining_email_2004} is based on the fact that legitimate SMTP servers have implemented the retry mechanism as required by SMTP,
			but a spammer seldom retries if sending fails.
		It usually works with a greylist that records the failed messages and the MTAs failed on their first tries.

		\textbf{Cryptographic:}
		Pretty Good Privacy (PGP) \cite{pgp_2007} and S/MIME are both cryptographic approaches that sign the message body using public-key cryptography and append the signature in the body.
		In PGP,
			Keys are stored in end-user key rings or in public key-servers.
		Key management uses a peer-to-peer web of trust architecture.
		Whereas in S/MIME,
			management follows a hierarchical model similar to SSL and keys are signed by a certificate authority.

\paragraph{Seder-based Behavior}

		\textbf{Delaying:} As a variation of rate limiting,
			delaying is triggered by an unusually high sending rate.
		Most delaying mechanisms,
			such as \cite{diego_distributed_2003} are applied at receiving MTAs.

		\textbf{Sender Behavior Analysis:} This technique examines behavior of incoming SMTP connections to distinguish spam from normal email.
		Messages from the machine exhibiting characteristics of malicious behavior such as directory harvest are blocked before reaching mailbox (e.g. postini).

%\end{description}

\subsubsection{Sender-oriented techniques}

To effectively deny spam at the source,
	ISPs and ESPs (Email Service Providers) have taken various measures to manage the usage of email services.
For example,
	message submission protocol \cite{BibEntry1998Dec} has been proposed to replace SMTP,
	when a message is submitted from an MUA (Mail User Agent) to its MTA.
	
%COAT
The proposed work in \cite{papasratorn_coat:_2012} differs from the other techniques in a way that all of them categorize mail messages at receiver side,
	whereas COAT works at the sender side and reduces outgoing spam rather than inbox spam.
%We have hardly found any work in literature about saving the Internet bandwidth and resource wastage by spam.

\textbf{Cost-based approaches:} Borrowing the idea of postage from regular mail systems,
	many cost-based techniques attempt to shift the cost of thwarting spam from receiver side to sender side.

All these techniques assume that the average email cost for a normal user is trivial and negligible,
	but the accumulative charge for a spammer will be high enough to drive them out of business.

Cost concept may have different forms in different proposals.
Bonded Sender \cite{BibEntry2018May} advocates associating email with real money,
	while SHRED \cite{krishnamurthy_shred:_2004} proposes affixing electronic stamps to messages.
Both centralized \cite{krishnamurthy_shred:_2004} and distributed \cite{walsh_distributed_2006} cost enforcement mechanisms have also been proposed.

%\subsubsection{Web (HTTP)}

%%Trust and Privacy Enabled Service Composition using Social Experience
%Trust aware approaches for Web service composition have been investigated widely in the literature.

%%1 reputation based approach
%Kuter and Golbeck [1] targeted OWL-S upper ontology and followed a reputation based approach for selecting highly trusted composite web service.

%%2 multi-agent based reputation model
%Paradesi et al. [15] adopted a multi-agent based reputation model to define trustworthiness of services.
%Moreover,
%	they developed a trust framework to derive trust for a composite service from trust model of component services.

%3

%4

%5

%6

%7

%8

%9

%10


