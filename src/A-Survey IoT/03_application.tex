\section{By Paquet}

\subsection{CoAP}

\begin{figure}[h]
	\footnotesize
	\centering
	\begin{bytefield}[bitwidth=1em]{32}
		\bitheader{0-31}                                                     \\
		\y{2}{Ver} & \y{2}{T} & \y{4}{TKL} & \y{8}{Code} & \y{16}{Message ID}\\
		\y{32}{Token}                                                        \\
		\y{32}{Options}                                                      \\
		\y{8}{11111111}  \y{24}{Payload}                                     \\
	\end{bytefield}
\caption{CoAP Header.}\label{fig:2}
\end{figure}


\begin{itemize}
	\item \textbf{Ver:}  is the version of CoAP
	\item \textbf{T:}  is the type of Transaction
	\item \textbf{TKL:}  Token length
	\item \textbf{Code:} represents the request method (1-10) or response code (40-255).
		\begin{itemize}
			\item Ex: the code for GET, POST, PUT, and DELETE is 1, 2, 3, and 4, respectively.
		\end{itemize}
	\item \textbf{Message ID:} is a unique identifier for matching the response.
	\item \textbf{Token:}  Optional response matching token.
\end{itemize}
	
\subsection{MQTT}

\begin{figure}[h]
	\footnotesize
	\centering
	\begin{bytefield}[bitwidth=4em]{8}
		\bitheader{0-7}                                                     \\
		\y{4}{Message Type} & \y{1}{UDP} & \y{2}{QoS Level} & \y{1}{Retain}  \\
		\y{8}{Remaining length}                                                        \\
		\y{8}{Variable length header}                                                      \\
		\y{8}{Variable length message payload}                                     \\
	\end{bytefield}
\caption{CoAP Header.}\label{fig:2g}
\end{figure}


\begin{itemize}
\item \textbf{Message type:}  CONNECT (1), CONNACK (2), PUBLISH (3), SUBSCRIBE (8) and so on
\item \textbf{DUP flag:}  indicates that the massage is duplicated
\item \textbf{QoS Level:} identify the three levels of QoS for delivery assurance of Publish messages
\item \textbf{Retain field:}  retain the last received Publish message and submit it to new subscribers as a first message
\end{itemize}

\subsection{XMPP}

\begin{itemize}
	\item Extensible Messaging and Presence Protocol
	\item Developed by the Jabber open source community
	\item An IETF instant messaging standard used for:
	\begin{itemize}
		\item multi-party chatting, voice and telepresence
	\end{itemize}
	\item Connects a client to a server using a XML stanzas
	\item An XML stanza is divided into 3 components:
	\begin{itemize}
		\item message: fills the subject and body fields
		\item presence: notifies customers of status updates
		\item iq (info/query): pairs message senders and receivers
	\end{itemize}
	\item Message stanzas identify:
	\begin{itemize}
		\item the source (from) and destination (to) addresses
		\item types, and IDs of XMPP entities
	\end{itemize}
\end{itemize}

\subsection{AMQP}

\begin{itemize}
	\item \textbf{Size:} the frame size.
	\item \textbf{DOFF:} the position of the body inside the frame.
	\item \textbf{Type:} the format and purpose of the frame.
	\begin{itemize}
		\item Ex: 0x00 show that the frame is an AMQP frame
		\item Ex: 0x01 represents a SASL frame.
	\end{itemize}
\end{itemize}

\subsection{DDS}

\begin{itemize}
	\item Data Distribution Service
	\item Developed by Object Management Group (OMG)
	\item Supports 23 QoS policies:
	\begin{itemize}
		\item like security, urgency, priority, durability, reliability, etc
	\end{itemize}
	\item Relies on a broker-less architecture
	\begin{itemize}
		\item uses multicasting to bring excellent Quality of Service
		\item real-time constraints
	\end{itemize}
	\item DDS architecture defines two layers:
	\begin{itemize}
		\item \textbf{DLRL:} Data-Local Reconstruction Layer
		\begin{itemize}
			\item serves as the interface to the DCPS functionalities
		\end{itemize}
		\item \textbf{DCPS:} Data-Centric Publish/Subscribe
		\begin{itemize}
			\item delivering the information to the subscribers
		\end{itemize}
	\end{itemize}
	\item 5 entities are involved with the data flow in the DCPS layer:
	\begin{itemize}
		\item Publisher:disseminates data
		\item DataWriter: used by app to interact with the publisher
		\item Subscriber: receives published data and delivers them to app
		\item DataReader: employed by Subscriber to access received data
		\item Topic: relate DataWriters to DataReaders
	\end{itemize}
\end{itemize}
\begin{itemize}
	\item No need for manual reconfiguration or extra administration
	\item It is able to run without infrastructure
	\item It is able to continue working if failure happens.
	\item It inquires names by sending an IP multicast message to all the nodes in the local domain
	\begin{itemize}
		\item Clients asks devices that have the given name to reply back
		\item the target machine receives its name and multicasts its IP @
		\item Devices update their cache with the given name and IP @
	\end{itemize}
\end{itemize}

\subsection{mDNS}

\begin{itemize}
	\item Requires zero configuration aids to connect machine
	\item It uses mDNS to send DNS packets to specific multicast addresses through UDP
	\item There are two main steps to process Service Discovery:
	\begin{itemize}
		\item finding host names of required services such as printers
		\item pairing IP addresses with their host names using mDNS
	\end{itemize}
	\item Advantages
	\begin{itemize}
		\item IoT needs an architecture without dependency on a configuration mechanism
		\item smart devices can join the platform or leave it without affecting the behavior of the whole system
	\end{itemize}
	\item Drawbacks
	\begin{itemize}
		\item Need for caching DNS entries
	\end{itemize}
\end{itemize}







%\newgeometry{top=2.5cm, bottom=2cm, left=.3cm, right=.3cm}
	
% \begin{table}[h!]
% \scriptsize
% \centering
% 	\begin{tabular}{lllll}
% 	Year &                            & \textbf{Factors}                  & \textbf{Computation Model} & \textbf{Results interpretation}                               \\\hline
% 	2018 & \cite{thubert_6tisch_2015} & -Closeness Centralityjhjhjhjhjhjh & Estimation                 & \textbf{Closeness} have a high degree of nbnbnbnbnbnbnb       \\
% 	\    &                            & -Degree Centrality                &                            & correlation with \textbf{privacy score}                       \\\hline
% 	\end{tabular}
% 	\caption{\label{tab:Table} An example table.}
% \end{table}

%\restoregeometry

