\section{Results exploitation} \label{sec:Results exploitation}

%%%%%%%%%%%%%%%%%%%%%%%%%%%%%%%%%%%%%%%%%%%%%%%%%%%%%%%%%%%%%%%%%% Initiation %%%%%%%%%%%%%%%%%%%%%%%%%%%%%%%%%%%%%%%%%%%%%%%%%%%%%%%%%%%%%%%%%%%%%%%%%
Below we report the results of applying the contagion process model to the Enron Email dataset.

%%%%%%%%%%%%%%%%%%%%%%%%%%%%%%%%%%%%%%%%%%%%%%%%%%%%%%%%%%%%%%%%%%%%%%%%%% Figure 1 %%%%%%%%%%%%%%%%%%%%%%%%%%%%%%%%%%%%%%%%%%%%%%%%%%%%%%%%%%%%%%%%%%%
\Figure{h!}{1}{cdf.png}{Cumulative distribution function of infected users}

Figure \ref{fig:cdf.png} shows the cumulative distribution function of the vulnerability contagion process,
	it appears clearly that the vulnerability diffusion process increases as the reputation level of vulnerable users increases,
	because when the vulnerability contagion process is at its $7^{th}$ iteration,
	the cumulative distribution function with the highest user's reputation level (0.8) is 1,
	this means that after the $7^{th}$ iteration,
	all users are infected.
This is not the case of users with low reputation level which need further contagion steps to diffuse their vulnerability widely in the network.
As a consequence,
	we can say that vulnerability contagion process speed is highly correlated with users reputation,
	it infects a large number of users quickly when the user's reputation level tends to 1.

%%%%%%%%%%%%%%%%%%%%%%%%%%%%%%%%%%%%%%%%%%%%%%%%%%%%%%%%%%%%%%%%%%%%%%%%%% Figure 2 %%%%%%%%%%%%%%%%%%%%%%%%%%%%%%%%%%%%%%%%%%%%%%%%%%%%%%%%%%%%%%%%%%
\Figure{h!}{1}{contagion.png}{Contagion process convergence}

Figure \ref{fig:contagion.png} shows the convergence of the vulnerability contagion process,
	the mean distance between users' social vulnerability scores in each iteration is calculated to see the convergence of the process.
The convergence process shows a high convergence level when the reputation coefficient is 0.8.
This happens when the mean distance between the users' social vulnerability scores calculated at each iteration still the same.
In other words,
	there are no more users to affect and the contagion process is at its high level.
After viewing these tow graphs,
	we can conclude that over-trusting vulnerable users allow them to get a high reputation level and consequently infects the entire social vulnerability scores.

%%%%%%%%%%%%%%%%%%%%%%%%%%%%%%%%%%%%%%%%%%%%%%%%%%%%%%%%%%%%%%%%%%%%%%%%%% Figure 3 %%%%%%%%%%%%%%%%%%%%%%%%%%%%%%%%%%%%%%%%%%%%%%%%%%%%%%%%%%%%%%%%%%
\FigureH{h}{0.45}{local_.png}  {Individual privacy vulnerability}
					{social_.png} {Social privacy vulnerability}
					{graph}       {Individual \& Social privacy vulnerabilities}

The initial and final measures of the simulation are represented in Figures \ref{fig:graph}.
Figure \ref{fig:local_.png} shows the initial privacy scores,
	these scores are generated randomly to illustrate the user's individual privacy vulnerability without caring about their friends' vulnerability.
Figure \ref{fig:social_.png} shows the final social privacy scores of the contagion process,
	these scores reveal the social vulnerability of users.
Users with dark color are more vulnerable than others with light color in terms of friendship with other users.
As a consequence,
	they have a high level of social vulnerability scores.

%%%%%%%%%%%%%%%%%%%%%%%%%%%%%%%%%%%%%%%%%%%%%%%%%%%%%%%%%%%%%%%%%%%%%%%%%% Table 1 %%%%%%%%%%%%%%%%%%%%%%%%%%%%%%%%%%%%%%%%%%%%%%%%%%%%%%%%%%%%%%%%%%
\Table{|c|c|c|}{table1}{Difference between Individual \& Social privacy vulnerabilities}{
	\ User ID & Individual vulnerability & Social vulnerability \\\hline
	\ 34      & 0.84                     & 0.67                 \\
	\ 67      & 0.12                     & 0.87                 \\
	\ 206     & 0.76                     & 0.33                 \\
	\ 588     & 0.23                     & 0.78                 \\\hline
}

Table \ref{table:table1} illustrates the input (Individual vulnerability) and the output (Social vulnerability) values of four arbitrary users of our dataset.
Results show that users with low individual vulnerability (e.g. user 67) could have a high social vulnerability due to their interactions with vulnerable users.
In contrast,
	users with high individual vulnerability (e.g. user 206) can, in turn, be less vulnerable from interactions with other users,
	but harms considerably the social vulnerability of their interlocutors.

%%%%%%%%%%%%%%%%%%%%%%%%%%%%%%%%%%%%%%%%%%%%%%%%%%%%%%%%%%%%%%%%%%%%%%%%%% Final findings %%%%%%%%%%%%%%%%%%%%%%%%%%%%%%%%%%%%%%%%%%%%%%%%%%%%%%%%%%%%%
In summary,
	results presented in this section show that if the trust coefficient between users is up to 0.8,
	the vulnerability diffusion process through trust relationship is at its high level of speed.
This what happens when a new information appears in a communication network and users forward it largely in the network.
In addition,
	this work gives a new insight to understand the relationship between trust,
	reputation,
	individual vulnerability and social vulnerability in the context of messaging services such as emails.


