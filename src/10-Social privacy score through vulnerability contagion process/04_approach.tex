\section{Approach} \label{sec:Approach}

The aim of this study is to understand how trust coefficient affects social privacy vulnerability.
In this section,
	we present our vulnerability contagion process in messaging services in order to get social vulnerability scores of users.
These scores represent the vulnerability within the social environment of each user.
In our study,
	we focus on the impact of trust and reputation in spreading out the privacy vulnerability.

To model this process we measure the impact of vulnerable users in protecting friends privacy.
For example,
	let us say that a user is exchanging messages with five friends as shown in Figure \ref{fig:user},
	the reputation of this user is given as the probability to be trusted by his friends (Figure \ref{fig:p_trust.png}).
The more trustworthy a user is,
	the more reputed he becomes.
However,
	if a user with a high reputation level has a high vulnerability,
	he can spread his vulnerability with a high infection coefficient (Figure \ref{fig:p_infection.png}).
As a consequence,
	the social vulnerability of users is calculated as the level of the contagion degree of each user in the communication network.
To get these values,
	a weighted matrix value M is used as an adjacency matrix normalized by users' degree.
Social vulnerability is calculated in a continuous space,
	this means that we iterate the vulnerability contagion process until the process converges as shown in Figure \ref{fig:contagion2.png}.

The probability of the infection is based on the trust level between users.

\FigureH{h}{0.44}{p_trust.png}{Trust grant}{p_infection.png}{Vulnerability contagion process}{user}{Reputation coefficient}

To evaluate the impact of trust in this process we add a reputation parameter $p_{reputation}$,
	this parameter is used to know how likely a user could be trusted by his friends,
	it is calculated as the probability to get at least one trust grant from them.

\Equation{1}{p_{reputation} = p(X \ge 1)~=~1~-~(1~-~p_{trust})^{n}}

Where,
	\Itemize{
	\item $X$:~Number of trust grant from friends,~$X \sim B(n,p)$.
	\item n = number of user's friends.
	\item $p_{trust} = p(X=1)$:~Probability to get one trust grant from a friend.
}

\Figure{h!}{.7}{contagion2.png}{Contagion \& peer influence example}

Trust parameter,
	in this equation,
	is added as a coefficient parameter to increase or decrease the vulnerability contagion process.
Whether a user can infect other users social vulnerability scores depends on the trust level between them,
	so users should distrust vulnerable users to preserve their own privacy and the privacy of the entire communication network.

The number of friends infected by each user u in each step of the vulnerability contagion process is given as:

\Center{$new~infected = old~infected + p_{reputation} * degree(u)$}

Initial privacy vulnerability scores of each user are given as input to our algorithm to estimate social privacy vulnerability scores,
	a normal distribution is used to generate initial privacy vulnerability scores.
The vulnerability contagion equation is given as:

\Equation{1}{P_{i+1} =  p_{reputation} * (M * P_{i}) + (1~-~p_{reputation}) * P_{i}}

Where,
	\Itemize{
	\item $P_{0}$ is the initial individual privacy vulnerability scores of each user.
	\item M is the adjacency matrix normalized by users' degree.
	\item i is the diffusion process iteration level.
}

The first part of this equation computes the vulnerability of a user based on his friends' average vulnerabilities weighted by the trust level with them.
The second part computes the vulnerability of a user based only on his own vulnerability weighted by his friends' distrust level.
Social vulnerability is a function of friends vulnerability and the trust coefficient.
Mean distances between privacy vulnerability scores of each iteration is calculated to get the convergence process shown in Figure \ref{fig:contagion.png}.

%\subsection{Updating reputation} \cite{paradesi_integrating_2009}
%Survey on trust models:
%	\cite{jiang_understanding_2016}\cite{aydillo_trustware_2015}

%Aspects of trust \cite{khaksari_tpta_2017}



