\section{Related work} \label{sec:Related work}

Transmission parameter configuration mechanisms,
	such as \ac{ADR} scheme \cite{LorawanSpecification} need to be developed to fit each application requirement in terms of power consumption,
	delay and packet delivery ratio.
Solutions running on LoRa node \red{should be less complex to match computation limitation of \ac{IoT} devices} as required in LoRaWAN specification.
However,
	LoRa network server could run complex management mechanism,
	which can be developed to improve network performance.
In this paper, 
	we focus on the server-side mechanisms.

The basic \ac{ADR} scheme \cite{LorawanSpecification} provided by LoRaWAN predicts channel conditions using the maximum received \ac{SNR} in the last 20 packets.
The basic \ac{ADR} scheme is sufficient when the variance of the channel is low, 
	it reduces the interference compared with the static data rate \cite{bor_lora_2016}\cite{slabicki_adaptive_2018}.
However,
	their simplicity causes many potential drawbacks.
First,
	the diversity of LoRa Gateway models that measures \ac{SNR} make the measurement inaccurate as a result of hardware calibration and interfering transmissions.
Second,
	selecting the maximum \ac{SNR} each 20 packets received could be a very long period in many IoT applications that require less uplink transmission.
Third,
	transmission parameters adjustment considers only the link of a single node.
If many LoRa nodes are connected to the near gateway,
	all nodes connected to this \red{one} will use the fastest data rate.
In this case,
	the number of LoRa nodes using the same data rate will increase and the probability of collisions also increases dramatically.

%The most common metrics used in the literature are \ac{SNR} and \ac{RSSI} to control \ac{Tx} and \ac{SF}.
For example,
	the authors in \cite{slabicki_adaptive_2018} slightly modify the basic \ac{ADR} scheme by replacing the maximum \ac{SNR} with the average function.
In this paper,
	we focus on building a framework that help IoT devices to adapt their transmission parameters to the application requirements in a server side.
% In \cite{cuomo_explora_2017},
% 	EXPLoRa-SF selects spreading factors based on the total number of connected nodes and EXPLoRa-AT equalizes the time-on-air of packets transmitted at the different spreading factors.
% In addition,
% 	the authors in \cite{bor_lora_2017} propose a link probing regime to select transmission parameters in order to achieve lower energy consumption.
% In \cite{reynders_power_2017},
% 	the authors present a scheme to optimize the packet error rate fairness to avoid near-far problems.

% Some seminal papers on LoRaWAN such as \cite{petajajarvi_coverage_2015b},
% 	\cite{wixted_evaluation_2016} test the coverage range and packet loss ratio by means of empirical measurements,
% 	but without investigating the impact of the parameters setting on the performance.

% Other works,
% 	such as \cite{bor_lora_2017},
% 	examine the impact of the modulation parameters on the single communication link between an ED and its GW,
% 	without considering more complex network configurations.
% 2 Note that,
% 	from the GW perspective,
% 	ACK packets are not distinguishable from any other DL packet and,
% 	hence,
% 	are subject to the same rules and constraints.

% To obtain more general results,
% 	\cite{li_2d_2016} uses a stochastic geometry model to jointly analyze interference in the time and frequency domains.
% It is observed that when implementing a packet repetition strategy,
% 	i.e.,
% 	transmitting each message multiple times,
% 	the failure probability reduces,
% 	but clearly the average throughput decreases because of the introduced redundancy.

% In \cite{ferre_collision_2017a} the author proposes closed-form expressions for collision and packet loss probabilities and,
% 	under the assumption of perfect orthogonality between SFs,
% 	it is shown that the Poisson distributed process does not accurately model packet collisions in LoRaWAN.
% Network throughput,
% 	latency and collision rate for uplink transmissions are analyzed in \cite{sorensen_analysis_2017} that,
% 	using queueing theory and considering the Aloha channel access protocol and the regulatory constrains in the use of the different sub-bands,
% 	points out the importance of a clever splitting of the traffic in the available sub-bands to improve the network performance.

% In \cite{bankov_mathematical_2017a} the authors present a mathematical model of the network performance,
% 	taking into account factors such as the capture effect and a realistic distribution of SFs in the network.
% However,
% 	the model does not include some important network parameters,
% 	preventing the study of their effect on the network performance.

% A step further is made in \cite{capuzzo_mathematical_2018} where the authors develop a model that makes it possible to consider various parameters configurations,
% 	such as the number of ACKs sent by the GW,
% 	the SF used for the downlink transmissions,
% 	and the DC constraints imposed by the regulations.
% In this work,
% 	however,
% 	multiple retransmissions have not been considered.

% The study presented in \cite{pop_does_2017} features a system-level analysis of LoRaWAN,
% 	and gives significant insights on bottlenecks and network behavior in presence of downlink traffic.
% However,
% 	besides pointing out some flaws in the design of the LoRaWAN medium access scheme,
% 	this work does not propose any way to improve the performance of the technology.


% In \cite{vandenabeele_scalability_2017},
% 	system-level simulations are again employed to assess the performance of confirmed and unconfirmed messages and show the detrimental impact of confirmation traffic on the overall network capacity and throughput.
% Here,
% 	the only proposed solution is the use of multiple gateways,
% 	without deeply investigating the specificities of the LoRaWAN standard.


% In \cite{reynders_lorawan_2018} a module for the ns-3 simulator is proposed and used for a similar scope,
% 	comparing the single and multi gateway scenarios and the use of unconfirmed and confirmed messages.
% In this case,
% 	the authors correctly implement the GW’s multiple reception paths,
% 	but do not take into account their association to a specific UL frequency,
% 	which usually occurs during network setup:
% 	indeed,
% 	the number of packets that can be received simultaneously on a given frequency can not be greater than the number of reception paths that are listening on that frequency.
% Also in this case,
% 	the study only focuses on the performance analysis,
% 	without proposing any improvement.


% The authors in \cite{hauser_proposal_2017},
% 	\cite{slabicki_adaptive_2018} target the original \ac{ADR} algorithm proposed by \cite{thethingsnetwork},
% 	suggesting possible ameliorations.
% Generally,
% 	the modified algorithms yield an increase of network scalability,
% 	fairness among nodes,
% 	packet delivery ratio and robustness to variable channel conditions.



% In \cite{reynders_power_2017},
% 	the authors compute the optimal SFs distribution to minimize the collision 4 probability and propose a scheme to improve the fairness for nodes far from the station by optimally assigning SFs and transmit power values to the network nodes,
% 	in order to reduce the packet error rate.



% In \cite{kouvelas_employing_2018} it is shown how the use of a persistent-Carrier Sense Multiple Access (p-CSMA) MAC protocol when transmitting UL messages can improve the packet reception ratio.
% However,
% 	attention must be payed to the fact that having many EDs that defer their transmission because of a low value of p may lead to channel under-utilization.



% In \cite{zucchetto_uncoordinated_2017},
% 	the authors investigate,
% 	via simulation,
% 	the impact of \ac{DC} restrictions in LPWAN scenarios,
% 	showing that rate adaptation capabilities are indeed pivotal to maintain reasonable level of performance when the coverage range and the cell load increase.
% However,
% 	the effect of other parameters setting on the network performance is not considered.



