\subsection{Experimentation}

\begin{frame}{Contribution}{Contributions}
	\Columns{0.5}{0.5}{
		\begin{itemize}
			\item \textbf{Use cases (Application Requirements)}
			\begin{itemize}
				\item Smart building: Voice, Images, Text. 
				% \item Smart traffic: Voice, Images, Text
			\end{itemize}
		\end{itemize}

		\begin{itemize}
			\item \textbf{Environments}
			\begin{itemize}
				\item Rural/Urban
				\item Static/Mobile
				\item Temperature
			\end{itemize}
		\end{itemize}

		\begin{itemize}
			\item \textbf{Scenarios}
			\begin{itemize}
				\item Application protocol (MQTT, COAP, XMPP)
				\item Network protocol (Start, Mesh)
				\item MAC protocol (LoraWan, Sigfox, ...)
			\end{itemize}

			\item \textbf{Input:}
			\begin{itemize}
				\item Service QoS metrics requirements
				\item MAC configuration (SF, CR, BW, ...)
				\item Network QoS metrics
			\end{itemize}
		\end{itemize}
	}{
	\begin{itemize}
		\item \textbf{Algorithms:}
		\begin{itemize}
			\item MADM
			\begin{itemize}
				\item Ranking methods
				\item Ranking \& weighted methods
			\end{itemize}
			\item Game theory
			\begin{itemize}
				\item Users vs users
				\item Users vs networks
				\item Networks vs network
			\end{itemize}
			\item Fuzzy logic
			\begin{itemize}
				\item as a score method
				\item another theory
			\end{itemize}
			\item Utility function
			\begin{itemize}
				\item 1
				\item 2
			\end{itemize}
		\end{itemize}

		\item \textbf{Outputs:}
		\begin{itemize}
			\item Ranked networks
		\end{itemize}
	\end{itemize}
}
\end{frame}

\begin{frame}{Technical choice}{Implementation}

	\Columns{0.65}{0.35}{
	\begin{itemize}
		\item ZOLERTIA RE-MOTE
			\begin{itemize}
				\item Low consumption component
				\item ADC port for placing sensors on it
			\end{itemize}
			
		\item CONTIKI OS
			\begin{itemize}
				\item Operating system for wireless and low power development
				\item Support for newer standards (6LowPAN, RPL, CoAP, MQTT) 
			\end{itemize}
			
		\item 6LowPAN
			\begin{itemize}
				\item Based on IPv6 and IEEE 802.15.4
				\item IPv6-based network with low power consumption
				\item Ability to create a mesh network
			\end{itemize}
			
		\item Sending packages
			\begin{itemize}
				\item UDP in the 6LowPAN network
				\item MQTT between the cloud platform and the router
			\end{itemize}
	\end{itemize}
		
	}{
	}
\end{frame}



\begin{frame}{Experimentation}{Experimentation}

	\Columns{0.65}{0.35}{
		\Itemize{
			\item a
			\item b
		}
	}{
		\Figure{!htb}{1}{mail.png}{}
	}
\end{frame}

%	\note{
%		\begin{itemize}
%			\item Objectifs
%			\begin{itemize}
%				\item Prouver la faisabilité de l'approche
%				\item Valider certains choix dans l’approche
%				\item Configurer l’approche
%				\item Tester l’approche dans des conditions extrêmes
%				\item Comparer l’approche par rapport à l’existant
%			\end{itemize}
%			\item Etapes
%			\begin{itemize}
%				\item Établir un plan d’expérimentation
%				\item Préparer un jeux de données
%				\item Préparer des scénarii
%				\item Dérouler les scénarii
%			\end{itemize}
%			\item Conseils:
%			\begin{itemize}
%				\item Utiliser des benchmarks de préférence
%				\item Montrer par rapport à la problématique les situations où l’approche est intéressante et là où elle ne l’est pas
%				\item Présenter les résultats sous forme de graphes
%				\item Bien expliquer et analyser les résultats
%			\end{itemize}
%		\end{itemize}
%	}


