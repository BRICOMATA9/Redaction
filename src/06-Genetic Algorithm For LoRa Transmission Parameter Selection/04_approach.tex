\section{Proposed Framework} \label{sec:Approach}


% A generic scheme to solve the configuration selection problem and any other similar selection problem is given in this section..
% The genetic selection scheme consists of three main steps,
% 	the first step contains a set of small parallel fuzzy logic (FL)-based subsystems,
% 	the second step is a multiple criteria decision making (MCDM) system,
% 	and the third step is a genetic algorithm (GA)-based component to assign a suitable weight for the criteria in the second component.
% The scheme decision phase can be described in more detail as follows.


% \begin{figure}
% \placetextbox{0.75}{0.8}{
% 		\tiny \mywhiteblackbox{
% 		\begin{tabular}{l} 
% 			Source program \\\hline
% 	        Ada\\
% 	        C/C++\\
% 	        Java\\ 
% 		    Perl\\
% 		    Python\\
% 		    ...
% 	    \end{tabular}
% 	}
% }
% \end{figure}


% \Figure{h}{.5}{drawing.svg}{kjkjkj rd}
	% \begin{tikzpicture}
	% 	\scriptsize
	% 	% \node[draw,align=left,dashed, text width = 0.2\linewidth] at (3,6) {Criteria c1 fuzzy based control};

	% 	\node[draw,align=left,dashed, text width = .5cm, text height = .5cm] at (3,6) {Multiple criteria decision making};
	% 	\node[draw,align=left,dashed, text width = 2cm, text height = 2cm] at (6,6) {Multiple criteria decision making};
	% 	\node[draw,align=left,dashed, text width = 2cm, text height = 2cm] at (6,3) {Genetic algorithm to determine weights of criteria (w1, ..., wn)};
	% 	\draw[->,black,thick,dashed] (0,0) -- (1,1);
	% \end{tikzpicture}

% \textbf{Definition:} stopping criteria, population size P, and mutation probability pm\\
% \textbf{Generate} randomly the initial configurations \\
% \textbf{repeat:}\\
% . . . \textbf{for} each configuration do\\
% . . . . . . Train a model \& compute configuration's fitness\\
% . . . \textbf{end}\\
% . . . \textbf{for} each reproduction 1 ... P/2 do\\
% . . . . . . \textbf{Select:} 2 configurations based on fitness\\
% . . . . . . \textbf{Crossover:} Produce 2 child configurations\\
% . . . . . . \textbf{Mutate:} child configurations with pm\\
% . . . \textbf{end}\\
% \textbf{until} stopping criterion are met\\

% \State\textbf{Definition:} stopping criteria, population size P, and mutation probability pm\\
% \State\textbf{Generate} randomly the initial configurations


% \begin{algorithm}
% 	 \KwData{QoS constraints}
% 	 \KwResult{Ranked configuration list}



% \SetKwFunction{FMain}{Main}
% \SetKwProg{GA}{GA}{:}{end}
% \GA{}{
% 	\Repeat{stopping criterion is met}{
% 		\For{each configuration}{
% 			Train \& compute configuration's fitness
% 		}
% 		\For{each reproduction 1 ... P/2}{
% 			\textbf{Select:} 2 configurations based on fitness\;
% 			\textbf{Crossover:} Produce 2 child configurations\;
% 			\textbf{Mutate:} child configurations with pm\;
% 		}
% 	}
% }

% % \SetKwFunction{FMain}{Main}
% % \SetKwProg{FL}{FL}{:}{end}
% % \FL{}{
% %     \eIf{$error \geq e$}{
% %         Do that as well
% %     }{
% %         Do otherwise
% %     }
% %     \While{$something \not= 0$ }{	
% %         $var1 \leftarrow var2$  	
% %     }

% % }
% % \eIf{understand}{
% %   go to next section\;
% %   current section becomes this one\;
% % }{
% %   go back to the beginning of current section\;
% % }

% \caption{LoRa Transmission Parameter Selection}
% \end{algorithm}
\medskip

The scheme selection process can be described following these five steps:

\begin{enumerate}
	\item According to the Semtech SX1276 specification\cite{lorasemtech}, there is 6720 possible settings ($s_{1}$, ... ,$s_{6720}$) and the framework has to select the most optimal one or to rank them according to their relevance.
	\item The first step of the selection process depends on multiple criteria up to i ($c_{1}$, ... , $c_{i}$).
		Different type of criteria can be measured from different sources to cover the maximum point of views,
		as example,
		the network server requirements, the applications requirements and the devices conditions.
	\item The Fuzzy Logic (FL) based subsystem gives an initial score for each configuration that reflects its relevance.
		The different sets of scores ($d_{1}$, ... ,$d_{i}$) are sent to the \ac{MCDM} in the $5^{th}$ step.
	\item At the same time,
			the genetic algorithm (GA) \cite{alkhawlani_access_2008} assigns a suitable weight ($w_{1}$, ... ,$w_{i}$) for each initial selection decision according to the objective function that is required by the application.
			% according to the importance and sensitivities of ANS criteria to the different characteristics of a wireless heterogeneous environment.
	\item Using the initial scores coming from the $3^{rd}$ step and the weights that are assigned using the $4^{th}$ step,
			the multi criteria decision making{} \ac{MCDM} will select the most relevant settings and rank them according to their reward.
\end{enumerate}

\Figure{h}{1}{genetic}{The proposed scheme for LoRa transmission parameters selection based on \ac{GA}, \ac{FL} and \ac{MCDM}}
