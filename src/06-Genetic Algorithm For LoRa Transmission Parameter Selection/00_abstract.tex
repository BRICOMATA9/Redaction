\begin{abstract}

% Cause of the problem 1 ------------------------------------------------
The exponential grow of \red{\ac{IoT}} applications in both industry and academic research raises many questions in wireless sensor network.
% Cause of the problem 2 ------------------------------------------------
Heterogeneous networks of IoT devices strongly depend on the ability of IoT devices to adapt their data transmission parameters to each application requirement.
% The problem ------------------------------------------------
One of the most important problem of the emerging IoT networks is the limitation in terms of energy consumption and computation capability.
% Motivation to solve this problem ------------------------------------------------
This limitation could be addressed by using the edge computing to unload IoT devices from additional computation tasks.
Our work is motivated by the idea of matching each transmission configuration with a reward and cost values to satisfy applications constraints.
% Goal ------------------------------------------------
Our goal is to make IoT devices able to select the optimal configuration and send their data to the gateway with the QoS required by IoT applications.
% Challenges ------------------------------------------------
Determining the best configuration among 6720 settings is challenging.
The difficulty is mainly due to the lack of tools that could take all applications requirements into account to select the best settings.
% Approach 1 ------------------------------------------------
To address this problem,
	we use a genetic algorithm in an edge computing to select the transmission parameters needed by the application.
Each LoRa configuration represents a feature that needs to be selected to match better the QoS criteria.
% Approach 2 ------------------------------------------------
Particularly,
	we analyze the impact of selecting one configuration in 3 kinds of applications:
	text,
	voice and image transmission by modeling a new adaptive data rate selection process.
% Validation ------------------------------------------------
%We validate our approach by using both simulation and a real environment testbed.

% % Results1 ------------------------------------------------
% Simulation results show that over-trusting vulnerable users speeds the vulnerability diffusion process through the network.
% % Results2 ------------------------------------------------
% Furthermore,
% vulnerable users with high reputation level lead to a high convergence level of infection,
% this means that the vulnerability contagion process infects the biggest number of users when vulnerable users get a high level of trust from their interlocutors.

\Keywords{\red{Genetic algorithm; Fuzzy logic; LoRaWAN; Adaptive Data Rate (ADR).}}


\end{abstract}


