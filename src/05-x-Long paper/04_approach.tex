\section{Approach} \label{sec:Approach}

We use an accurate and realistic WSNet-based simulator [30] written in C/Modern C++ under CeCILL free license.
WSNet is a modular event-driven wireless network simulator that implements the required communication protocol layers and simulates the network behavior with a high level of accuracy.
We have extended the simulator in several aspects (e.g.,
	spectrum use,
	interference,
	capture effect) to take into account flexibility and specificity of the LoRa PHY/MAC layers.

\subsection{Spectrum}

To support PHY layer heterogeneity and flexibility,
	we have modified the core of the WSNet simulator by including a spectrum model.
This new core is inspired by Baldo et al.
[31] to provide a support for modeling the frequency-dependent aspects of communications.
Moreover,
	it exploits the spectrum in terms of spectral resources instead of logical channels.
It provides more accurate PHY models (several waveforms with different configurations) and a more accurate interference model for heterogeneous simulations.
Thus,
	we can evaluate different configurations of LoRa network (e.g.,
	SF,
	BW,
	channel) for different homogeneous and heterogeneous deployments.
Furthermore,
	it permits the evaluation of inter-technology interference (from a non LoRa technology),
	not presented in this article for space reasons.


\subsection{LoRa Modulation}

The LoRa modulation is an adaptation of the CSS modulation.
Its advantages are low power transmissions and channel robustness.
LoRa features seven orthogonal SFs from SF6 to SF12.
Symbol duration T s and bit rate R b are defined as follows:
%\equation


We have developed a LoRa modulation module in-
spired by the Equation 3 presented in [17]. This equation
shows BER as a function of SF and the energy per bit to noise ratio N
%equation

where Q(x) is the Q-function.




\subsection{Spreading Factors Orthogonality}
We have developed an interference module taking into account the effect of the quasi-orthogonality of SFs.
It uses the values for co-channel rejection for all combinations of the desired signal SF d and the interferer signal SF i [19] presented in Table I.

\begin{table}[h!]
\scriptsize
	\begin{tabulary}{\textwidth}{L|L|L|L|L}
	\  &  &  &  &  \\\hline
	\  &  &  &  &  \\\hline
	\  &  &  &  &  \\\hline
	\  &  &  &  &  \\\hline
	\  &  &  &  &  \\\hline
	\end{tabulary}
\caption{\label{tab:} }
\end{table}

\subsection{LoRa SX1276 and SX1301 Transceivers}

The Semtech SX1276 transceiver provides high interference immunity while minimizing energy consumption [7].
Sensitivity ρ dBm is defined according to the Semtech designer guide [5] as:
%equation

where −174 accounts for the thermal noise effect,BW is the receiver bandwidth,
	N F is the receiver noise factor for a given hardware implementation,
	and SN R is the minimum ratio of the desired signal power to noise that can be demodulated.
Table II shows sensitivity and the data rate for the SX1276 transceiver.


\begin{table}[h!]
\scriptsize
	\begin{tabulary}{\textwidth}{L|L|L|L|L}
	\  &  &  &  &  \\\hline
	\  &  &  &  &  \\\hline
	\  &  &  &  &  \\\hline
	\  &  &  &  &  \\\hline
	\  &  &  &  &  \\\hline
	\end{tabulary}
\caption{\label{tab:} }
\end{table}

The Semtech SX1301 offers breakthrough gateway capabilities with a multi-channel high performance transmitter/receiver designed to simultaneously receive several LoRa packets with different SFs and up to 8 channels [8].
It enables robust communications for a large number of nodes spread over a wide range.
In our simulator,
	we consider the gateway capture effect based on SX1301.
The gateway may receive,
	depending on the reception power,
	several packets with different SFs because they are quasi-orthogonal with respect to each other.
If the gateway receives two packets overlapped with the same SF,
	the gateway will rec

\subsection{MAC and Application Layers}

Our simulator considers a LoRa random access method that basically behaves like ALOHA.
The application layer manages the data traffic generation at each node inside a time window ”Application Period (AP)”.
At every AP occurrence,
	each node picks randomly a time instant inside the current window,
	and a data packet is generated and sent to the lower layer (network or MAC).
Hence,
	each node can generate only one data packet per AP.
We vary AP between 60 s and 40 min to evaluate performance for different application use cases.

\subsection{Deployment Scenarios}

In this subsection,
	we present the network deployment scenarios and the SF allocation strategy in each scenario.

\Itemize{
\item In SF i Homogeneous,
	nodes are uniformly deployed in a disk of radius equal to D max (SF i ),
	the maximum transmission range of SF i (Figure 1(a)),
	and select the same LoRa configuration SF i .

\item Multi-Homogeneous is the superposition of independent SF i homogeneous deployments (i from 6 to 12 and the corresponding radius D max (SF i )) with a single gateway at the center.
The number of nodes are equally distributed between homogeneous networks.
As each SF i homogeneous deployment is independent and uniform,
	we can see in Figure 1(b) higher density close to the gateway.

\item In Heterogeneous f(Dmax),
	nodes are uniformly deployed in a disk of radius equal to D max (SF 12 ),
	the maximum transmission range of SF12 and then,
	each node selects its configuration according to its link budget (i.e.,
	with the maximum data rate or the minimum SF).
As shown in Figure 1(d),
	some rings naturally appear.
Nodes in the farthest ring are configured with SF12,
	in the next ring with SF11,
	and so on until the last disk with SF6.

\item In Heterogeneous Random,
	nodes are still uniformly deployed in a disk of radius equal to D max (SF 12 ) but they randomly select their configuration among the ones available according to their link budget.
Hence,
	each node selects its LoRa configuration according to its budget link and its needs.
For example,
	nodes in the third ring can randomly select their SF between SF10 and SF12 depending on their optimization criteria (e.g.,
	energy consumption,
	data rate,
	reliability).
}

































