\section{Related work} \label{sec:Related work}

%%%%%%%%%% Research categories 
\subsection{LoRa Overview}
In this section,
	we give a short overview of the LoRa physical layer parameters.
LoRa specification [3],
	[4] and technical documents [5],
	[6],
	[7],
	[8] contain more detailed information.

\subsubsection{Bandwidth (BW):}
	is the range of frequencies available for transmission.
Larger BW increases the data rate but decreases sensitivity.
Typically,
	BWs are 125 kHz, 250 kHz,
	and 500 kHz.
However,
	Semtech SX1276 offers BW configurations from 7.8 kHz to 500 kHz.

\subsubsection{Carrier Frequency (CF):}
	is the central frequency in a band.
For our study,
	we consider the 868 MHz band.

\subsubsection{Coding Rate (CR):}
characterizes resilience to transmission errors.
Higher CRs result in better robustness,
	but increases the air time.
LoRa supports CR of 4/5, 4/6, 4/7, and 4/8.

\subsubsection{Spreading Factor (SF):}
is the ratio between the chip rate and the symbol rate.
For a given SF,
	there are 2 SF chips per symbol.
SF can be selected between 6 and 12,
	where SF12 presents the highest sensitivity and the longest range,
	but the lowest data rate.

\subsection{From Experimental Measurements to Simulations}
Some authors deployed LoRa networks and experimentally studied its performance [9] [10] [11] [12] [13].
The measurements were done in city centers,
	tactical troop tracking,
	and sailing monitoring systems.
Nevertheless,
	experimental results in real life networks are not reproducible and MAC layer optimization is difficult.
Blenn et al.
[14] performed simulations based on traces from experiments and analyzed results based on real life and large scale measurements from The Things Network but their simulations are limited to the deployed scenario.
To and Duda [15] presented LoRa simulations in NS-3 validated in testbed experiments.
They considered the capture effect and showed the reduction of the packet drop rate due to collisions with a CSMA approach.
In a system level simulator,
	Haxhibeqiri et al.
[16] studied the scalability for LoRaWAN deployments in terms of the number of nodes per gateway.
Simulations are performed for a duty cycle of 1\% but they are limited to 1000 nodes.
We developed a LoRa simulator to compare the performance in different deployment scenarios for large scale networks based on an accurate model of the LoRa PHY/MAC layers.
We simulate several deployment scenarios varying traffic intensity and the number of nodes.

\subsection{LoRa Evaluation and Limits}
Several authors evaluated performance and limits of LoRa networks.
Reynders et al.
[17] evaluated Chirp Spread Spectrum (CSS) and ultra-narrow-band networks.
They proposed a heuristic equation that gives Bit Error Rate (BER) for a CSS modulation as a function of SF and Signal to Noise Ratio (SNR).
Cattani et al.
[18] evaluated the impact of the LoRa physical layer settings on the data rate and energy efficiency.
They evaluated the impact of environmental factors such as temperature on the LoRa network performance and showed that high temperatures degrade the Packet Delivery Ratio (PDR) and Received Signal Strength (RSS).
Goursaud et al.
[19] studied the performance of the CSS modulation.
They showed the possibility of interference between different SFs and evaluated co-channel rejection for all combinations of SFs.
Feltrin et al.
[20] discussed the role of LoRaWAN for IoT and showed its application to many use cases.
They considered the effect of non perfect orthogonality of SFs for a link level analysis.
Petajajarvi et al.
[21] analyzed the scalability of a LoRaWAN wide area network and showed its good coverage (e.g.
until 30km on water for SF12 and transmit power of 14 dBm).
They also showed the maximum throughput for different duty cycles per node per channel.
Mikhaylov et al.
[22] discussed LoRa performance under European frequency regulations.
They studied the performance metrics of a single end device,
	then the spatial distribution of several end devices.
They showed LoRa strengths (large coverage and good scalability for low uplink traffic) and weaknesses (low reliability,
	delays,
	and poor performance of downlink traffic).
Bor et al.
[23] presented a capability and performance analysis of a LoRa transceiver and proposed LoRaBlink protocol for link-level parameter adaptation.
Nunez et al.
[24] analytically showed the potential gain of adaptive LoRa solutions that choose suitable radio parameters (i.e.,
	spreading factor,
	bandwidth,
	and transmission power) to different deployment topologies (i.e.,
	star and mesh).
These studies provide a first view of LoRa performance and its limitations.
As a conclusion we need to take into account the capture effect and imperfect orthogonality of SFs.
We contribute with an accurate LoRa simulation model considering the co-channel SF interference and the gateway capture effect,
	allowing accurate performance analysis in large scale simulations for different deployment scenarios.
Our study extends the previous evaluations of LoRa limits with the evaluation of reliability,
	network throughput,
	and power consumption from sparse to massive access deployment scenarios.

\subsection{LoRa Network Deployment Strategies}
Some authors studied LoRa network deployments and SF allocation strategies.
Bor et al.
[25] studied LoRa transceiver capabilities and the limit supported by LoRa system.
They showed that LoRa networks can scale if they use dynamic selection of transmission parameters.
Georgiou et al.
[26] investigated the effects of interference in a network with a single gateway.
They studied two link-outage conditions,
	one based on SNR and the other one based on co-SF interference.
They showed,
	as expected,
	that performance decreases when the number of nodes increases and highlighted the interest of studying spatially heterogeneous deployments.
Croce et al.
[27] showed the effect of the quasi-orthogonality of SFs and found that overlapped packet transmissions with different SFs may suffer from losses.
They validated the findings by experiments and proposed SIR thresholds for all combinations of SFs.
They remarked that LoRa networks cannot be studied as a superposition of independent networks because of imperfect SF orthogonality.
Abeele et al.
[28] studied the capacity and scalability of LoRaWAN for thousands of nodes per gateway.
They showed the importance of considering the capture effect and interference models.
They proposed an error model from BER simulations to determine communication ranges and interference.
They also analyzed three strategies of network deployments (random SF allocation,
	a fixed one,
	and according to related PDR),
	the last one presenting the best performance.
Lim et al.
[29] analyzed the LoRa technology to increase packet success probability and proposed three SF allocation schemes (equal interval based,
	equal area based,
	and random based).
They found that the equal area scheme results in better performance compared with other schemes because of the reduced influence of SFs.
The state of the art indicates the interest in heterogeneous deployments and SF allocation strategies.
Thus,
	we analyze homogeneous and heterogeneous deployments with different SF allocations as a function of the number of nodes and traffic intensity in order to show network performance and the benefits of heterogeneity for large scale networks.


%%%% Related work position 

%All previous work didn't take into consideration the topological aspect of interactions to measure social vulnerabilities of users.
%The closest study to our approach is that presented in \cite{zeng_trustaware_2014}.
%However,
%	this solution doesn't study the impact of having interactions with vulnerable users.
%In this paper,
%	we study the impact of trusting vulnerable users in preserving the privacy of all users in the communication network.




