\subsection{Related work}
\begin{frame}{Related work}{Comparison}
	\begin{table}
		\begin{tabular}{c|c|c|c|c}
			Paper & A1 & A2 & A3 & A4 \\\hline
					\cite{barro_lorawan_2019} &    &    &    & \\\hline
                                              &    &    &    & \\\hline
                                              &    &    &    & \\\hline
                                              &    &    &    & 
		\end{tabular}
		\caption{\label{tab:Tablej} An example table.}
	\end{table}
\end{frame}


%\note{
%	\begin{itemize}
%		\item Contenu:
%		\begin{itemize}
%			\item Tableau comparatif (articles connexes/avantages et désavantages)
%			\item Les limites de l’existant
%			\item Notre travaille traite le meme x que les travaux précidants mais utilise y au lieu de z (xy/xz)
%		\end{itemize}
%		\item Procedure:
%		\begin{itemize}
%			\item Lecture en largeur
%			\begin{itemize}
%				\item Lecture de beaucoup de papiers connexes
%				\item Comprendre le domaine
%				\item Comprendre les travaux existants
%				\item Sélection des travaux intéressants
%			\end{itemize}
%			\item Lecture en profondeur
%			\begin{itemize}
%				\item Lecture et analyse des travaux sélectionnés
%				\item Descendre jusqu’au détail du détail
%			%					\begin{itemize}
%			%						\item Poser toujours la question pourquoi?
%			%						\item Être capable d’implémenter de suite l’approche.
%			%					\end{itemize}
%				\end{itemize}
%				\item Situer le travail par rapport à l’existant sur la base de La problématique traitée
%				\begin{itemize}
%					\item Les critiques faites sur l’existant
%					\item Les hypothèses du travail courant
%					\item Les objectifs initiales du travail
%					\item Les résultats théoriques et expérimentales obtenus
%				\end{itemize}
%			\end{itemize}
%			\item Article:
%			\begin{itemize}
%				\item Est-ce que le problème est toujours intéressant ?
%				\item Est-ce qu'on peux traiter le problème d'une autre manière ?
%				\item Est-ce que les hypothèses sont réalistes ?
%				\item Est-ce que le travail est applicable dans le contexte actuel ?
%				\item Est-ce que tous les aspects du problème ont été traités ?
%				\item Existe-t-il d’autres manières pour le résoudre ?
%			\end{itemize}
%		\end{itemize}
%	}

\begin{frame}{Related work}{Comparison}
	\begin{table}
		\begin{tabular}{c|c|c|c|c}
			Paper & A1 & A2 & A3 & A4 \\\hline
				  &    &    &    & \\\hline
				  &    &    &    & \\\hline
				  &    &    &    & \\\hline
				  &    &    &    & 
		\end{tabular}
		\caption{\label{tab:Tableju} An example table.}
	\end{table}
\end{frame}

%\note{
%	\begin{itemize}
%		\item Conseils:
%		\begin{itemize}
%			\item Qu'est ce qui réuni et divise tous c'est travaux
%			\item Ce chapitre ne doit pas être une simple revue de la bibliographie
%			\item Présentation des travaux antérieurs et connexes
%			\begin{itemize}
%				\item Choisir les travaux reliés
%				\item Critique des travaux antérieurs
%				\item Description du lien entre le sujet traité dans le mémoire et les travaux antérieurs
%				\item le lien: qqch en commun (méthode, approche, outil ...)
%				\item Cibler les critiques où le candidat apporte des contributions
%				\item Résumer l’analyse de ces travaux dans un tableau récapitulatif
%				\item Il faut analyser les travaux pour proposer une contribution
%			%					\begin{itemize}
%			%						\item Classification des travaux
%			%						\item Avantages \& inconvénients
%			%						\item Contextes d’utilisation
%			%					\end{itemize}
%				\item Formulation du problème théorique
%				\item Présentation des hypothèses explicatives
%				\item Suite à la lecture de ce chapitre, le lecteur doit avoir compris la motivation pour le choix du sujet et son importance
%			\end{itemize}
%		\end{itemize}
%	\end{itemize}
%}

