\section{Introduction} \label{sec:Introduction}

% Needs: Context Current needs
Over the past few years,
	new approaches called Low Power Wide Area (LPWA) networking technologies have emerged.

These technologies became a new alternative to current generations of cellular networks (2G, 3G, and 4G),
	while covering large areas (typical range of 10 km).

Several papers have reviewed the different LPWA technologies.
For example,
	Raza et al. [1] surveyed standardization activities (IEEE, IETF, 3GPP, ETSI),
	as well as industrial ones around LPWA technologies (LoRa alliance, Weightless-SIG, Dash7 alliance).
They identified potential research directions to address limitations and challenges of LPWA technologies such as a massive number of devices,
	link optimization, and adaptability.

% Problematic Current bad state of the research
After giving an overview of LPWA and cellular technologies for IoT [2],
	the authors discussed the capabilities and limitations of LoRaWAN.
Nevertheless,
	the potential of an adaptive LoRa solution in terms of spreading factor,
	bandwidth,
	transmission power,
	and topology is still not well studied or exploited.
Thus,
	new protocols and strategies are required to improve LoRa scalability.

%Challenges
The first step is to understand heterogeneous network deployments when devices use different spreading factor.
In this paper,
	we investigate homogeneous (i.e.
when all the nodes select the same LoRa configuration) and heterogeneous deployments (i.e.
when each node selects its LoRa configuration according to its link budget or their needs) for a large number of devices (up to 10000 nodes per gateway).In order to evaluate performance,
	we have developed an accurate model of the PHY/MAC LoRa based on the extended WSNet simulator.

%Contribution 
The LoRa model takes into account spectrum usage,
	co-channel rejection due to quasi-orthogonality of the LoRa spreadspectrum modulation,
	and the gateway capture effect.
The simulation results give an insight on reliability,
	network capacity,
	and energy consumption for homogeneous and heterogeneous deployments as a function of the number of nodes and traffic intensity.

% The structure 
The article is organized as follows.
Section II gives a short overview of the LoRa technology and provides the state of the art on LoRa experimental measurements and simulations,
	LoRa limitations,
	and network deployment strategies as well as motivations.
Section III presents the developed LoRa WSNet based simulator and the deployment scenarios.
Section IV analyzes performance in terms of packet delivery ratio,
	throughput,
	and power consumption.
Section V concludes the article and gives some ideas for future work.




