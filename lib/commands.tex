%y
\def\eg{e.g.}
\def\ie{i.e.}
\def\refsec#1{Section \ref{sec:#1}}
\def\reffig#1{Figure \ref{fig:#1}}
\def\reftab#1{Table \ref{tab:#1}}
% \def\bf#1{\textbf{#1}}

\def\blue#1{\textcolor{beamer@blendedblue}{#1}}
\def\violet#1{\textcolor{violet}{#1}}
\def\yellow#1{\textcolor{yellow}{#1}}
\def\red#1{\textcolor{red}{#1}}
\def\green#1{\textcolor{green}{#1}}
\def\black#1{\textcolor{black}{#1}}
\def\shaded#1{\textcolor{gray!30}{#1}}
\def\d{\textcolor{red}{$\downarrow$}}
\def\u{\textcolor{green}{$\uparrow$}}
\def\ok{\textcolor{green}{\ding{51}}}
\def\ko{\textcolor{red}{\ding{55}}}
\def\mm{\blue{\textbf{-}}}

\def\Mark#1{\textsuperscript#1}
\def\idea#1{\textcolor{green}{#1}}

%addchapter
%\newcommand\addchapter[1]{
%	\addcontentsline{toc}{chapter}{#1}
%}

\newcommand{\symbolfootnote}{\renewcommand{\thefootnote}{\fnsymbol{footnote}}}
\newcommand{\normallinespacing}{\renewcommand{\baselinestretch}{1.5} \normalsize}
\newcommand{\mediumlinespacing}{\renewcommand{\baselinestretch}{1.2} \normalsize}
\newcommand{\narrowlinespacing}{\renewcommand{\baselinestretch}{1.0} \normalsize}



\usepackage{bytefield}
  \newcommand{\y}[2]{\bitbox{#1}{#2}}

% \usepackage{xparse}
% 	\ExplSyntaxOn
% 		\NewDocumentCommand{\xinput}{m}{%
% 			\azimut_xinput:n { #1 }
% 		}
% 		\ior_new:N \g_azimut_xinput_stream
% 		\cs_new_protected:Nn \azimut_xinput:n{
% 			\ior_open:Nn \g_azimut_xinput_stream { "|ls ~ #1" }
% 			\ior_map_inline:Nn \g_azimut_xinput_stream { \file_input:n { \tl_trim_spaces:n {##1} } }
% 		}
% 	\ExplSyntaxOff


%todo
\newcounter{todo}
\usepackage{tcolorbox}
	\newtcbox{\mytodobox}{colback=white,colframe=white!75!white}
\newcommand\todo[1]{
	\refstepcounter{todo}
	\mytodobox{\hypertarget{todo\thetodo}{#1}}
	\addcontentsline{tod}{subsection}{\protect\hyperlink{todo\thetodo}{\thetodo~#1}\par} }
\makeatletter
\newcommand\listoftodos{
	\@starttoc{tod}}
\makeatother
%\def\todo{\textcolor{red}{ToDO}}

\newcommand{\towFigure}[5]{
	\medskip
	\begin{figure}
		\includegraphics[width=#1\columnwidth]{#2} \\
		\includegraphics[width=#1\columnwidth]{#3}
		\caption{#5.}\label{fig:#4}
	\end{figure}
	\medskip
}

\newcommand{\towFigureT}[5]{
	\medskip
	\begin{figure}
		\includegraphics[width=#1\columnwidth]{#2} \\
		\includegraphics[width=#1\columnwidth]{#3}
		\caption*{\blue{Figure} \ref{fig:#4}: #5.}
%		\caption{#5.}\label{fig:#4}
	\end{figure}
	% \medskip
}

\newcommand{\Figure}[4]{
	% \medskip
	\begin{figure}[#1]
	\centering
	\includegraphics[width=#2\columnwidth]{#3}
	\caption{#4.}\label{fig:#3}
	\end{figure}
}

\newcommand{\Tickz}[4]{
	\medskip
	\begin{figure}[#1]
			\centering
			\begin{tikzpicture}[scale=#2,line width=1pt]
				\input{\PLOTPATH/#3}
			\end{tikzpicture}
	\caption{#4.}\label{fig:#3}
	\end{figure}
	\medskip
}

\newcommand{\FigureS}[4]{
	\medskip
	\begin{figure}[#1]
	\centering
	\includegraphics[width=#2\columnwidth]{#3}
	\caption*{\blue{Figure \ref{fig:#3}:} #4.}
	\end{figure}
	\medskip
}

\newcommand{\FigureT}[3]{
	\medskip
	\begin{figure}[#1]
	\centering
	\includegraphics[width=#2\columnwidth]{#3}
	\end{figure}
	\medskip
}


\newcommand{\FigureH}[8]{
%	\medskip
	\begin{figure}
		\centering
		\begin{subfigure}[#1]{#2\columnwidth}
			\centering
			\includegraphics[width=\columnwidth]{#3}
			\caption{#4.}\label{fig:#3}
		\end{subfigure}
		~ % \quad, \qquad, \hfill
		\begin{subfigure}[#1]{#2\columnwidth}
			\centering
			\includegraphics[width=\columnwidth]{#5}
			\caption{#6.}\label{fig:#5}
		\end{subfigure}
		
		\caption{#8.}\label{fig:#7}
	\end{figure}
%	\medskip
}

\newcommand{\FigureV}[8]{
%	\medskip
	\begin{figure}
		\centering
		\begin{subfigure}[#1]{\columnwidth}
			\centering
			\includegraphics[width=#2\columnwidth]{#3}
			\caption{#4.}\label{fig:#3}
		\end{subfigure}
		~ % \quad, \qquad, \hfill
		\begin{subfigure}[#1]{\columnwidth}
			\centering
			\includegraphics[width=#2\columnwidth]{#5}
			\caption{#6.}\label{fig:#5}
		\end{subfigure}
		
		\caption{#8.}\label{fig:#7}
	\end{figure}
%	\medskip
}

\newcommand{\TickzH}[9]{
%	\medskip
	\begin{figure}
	\begin{center}
		\begin{subfigure}[#1]{#2\columnwidth}
			\centering
			\begin{tikzpicture}[scale=#9,line width=1pt]
				\input{\PLOTPATH/#3}
			\end{tikzpicture}
			\caption{#4.}\label{fig:#3}
		\end{subfigure}
		~ % \quad, \qquad, \hfill
		\begin{subfigure}[#1]{#2\columnwidth}
			\centering
			\begin{tikzpicture}[scale=#9,line width=1pt]
				\input{\PLOTPATH/#5}
			\end{tikzpicture}
			\caption{#6.}\label{fig:#5}
		\end{subfigure}
		\caption{#8.}\label{fig:#7}
	\end{center}
	\end{figure}
%	\medskip
}

\newcommand{\TickzV}[8]{
%	\medskip
	\begin{figure}
		\centering
		\begin{subfigure}[#1]{\columnwidth}
			\centering
			\begin{tikzpicture}[scale=#2,line width=1pt]
				\input{\PLOTPATH/#3}
			\end{tikzpicture}
			\caption{#4.}\label{fig:#3}
		\end{subfigure}
		~ % \quad, \qquad, \hfill
		\begin{subfigure}[#1]{\columnwidth}
			\centering
			\begin{tikzpicture}[scale=#2,line width=1pt]
				\input{\PLOTPATH/#5}
			\end{tikzpicture}
			\caption{#6.}\label{fig:#5}
		\end{subfigure}
		
		\caption{#8.}\label{fig:#7}
	\end{figure}
%	\medskip
}


\newcommand{\Equation}[2]{
	\begin{equation}\label{eq:#1}
		#2
	\end{equation}
}

\newcommand{\EquationT}[2]{
	\begin{equation*}\tag{\ref{eq:#1}}
		#2
	\end{equation*}
}

\newcommand{\EquationS}[1]{
	\begin{equation*}
		#1
	\end{equation*}
}

\newcommand{\Itemize}[1]{
	\medskip
	\begin{itemize}
		#1
		\bigskip
	\end{itemize}
	\medskip
}

\newcommand{\Enumerate}[1]{
	\begin{enumerate}
		#1
		\bigskip
	\end{enumerate}
	\medskip
}

\newcommand{\Columns}[4]{
	\begin{columns}
		\begin{column}{#1\textwidth}
		#3
		\end{column}
		\begin{column}{#2\textwidth}
		#4
		\end{column}
	\end{columns}
}

\newcommand{\Table}[4]{
	\begin{table}[h!]
		\centering
		\begin{tabulary}{\textwidth}{#1}
			#4
		\end{tabulary}
		\caption{\label{table:#2} #3.}
	\end{table}
}

\newcommand{\TableT}[4]{
	\begin{table}[h!]
	\centering
		\begin{tabular}{#1}
			#4
		\end{tabular}
		\caption*{\blue{Table \ref{table:#2}:~} #3.}
	\end{table}
}



\newcommand{\compresslist}{%
	\setlength{\itemsep}{0pt}%
	\setlength{\parskip}{1pt}%
	\setlength{\parsep}{0pt}%
}



\AtBeginDocument{% ...if you're using hyperref
  \let\oldlabel\label% Copy original version of \label
  \let\oldref\ref% Copy original version of \ref
}
\newcommand{\addlabelprefix}[1]{%
  \renewcommand{\label}[1]{\oldlabel{#1-##1}}% Update \label
  \renewcommand{\ref}[1]{\oldref{#1-##1}}% Update \ref
}
\newcommand{\removelabelprefix}{%
  \renewcommand{\label}{\oldlabel}% Restore \label
  \renewcommand{\ref}{\oldref}% Restore \ref
}



% addab, addac
\usepackage{acro}
%first-long-format=\itshape
	\acsetup{hyperref=true}
	\newcommand{\addac}[2]{
		\DeclareAcronym{#1}{
			short = \ensuremath{#1},
			long  = {#2},
			sort  = {#1},
			class = nomencl
		}
	}
	\newcommand{\addab}[2]{
		\DeclareAcronym{#1}{
			short = {#1} ,
			long  = {#2},
			sort  = {#1},
			class = abbrev
		}
	}



















